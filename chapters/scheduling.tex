\section{Criteria}
Algorithms can be optimized for different goals:
\begin{itemize}
\item \textbf{CPU utilization}, keep CPU as busy as possible
\item \textbf{throughput}, number of processes that complete their execution per time unit
\item \textbf{turnaround time}, amount of time to execute a particular process
\item \textbf{waiting time}, amount of time a process has been waiting in the ready queue
\item \textbf{response time}, amount of time it takes from when a request was submitted until the first response is produce, not output

\end{itemize}

\section{Algorithms}
A detailed description is provided in Table \ref{Tab:Algorithms}.
\begin{table}[]
	\centering
	\resizebox{\textwidth/8*7}{!}{%
		\begin{tabular}{|m{\textwidth/11}|m{\textwidth/10}|m{\textwidth/11}|m{\textwidth/11}|m{\textwidth/9}|m{\textwidth/9}|m{\textwidth/9}|m{\textwidth/11}|m{\textwidth/7}|m{\textwidth/9}|m{\textwidth/8}|}
			\hline
			\textbf{Scheduler}                   & \textbf{Preemptive} & \textbf{Priority} & \textbf{CPU Utilization} & \textbf{Thrughput} & \textbf{Turnaround time}                             & \textbf{Waiting time}                           & \textbf{Response time}                          & \textbf{Advantages}                                                                                                                    & \textbf{Issues}                                                                                         & \textbf{Description}                                                                                                       \\ \hline
			First Come First Served (FCFS)       & NO                  & NO                & LOW                      & LOW                & SHORTEST                                             & HIGH                                            & HIGH                                            & Semplicity                                                                                                                             & Processes may starve if after long process                                                              & Favours CPU-bound processes                                                                                                \\ \hline
			Shortest Job First (SJF)             & NO                  & NO                & HIGH                     & HIGH               & LONG for long processes, SHORTEST for fast processes & HIGH for long processes, LOW for fast processes & HIGH for long processes, LOW for fast processes & Minimum Average Waiting time                                                                                                           & Long processes may starve, must estimate process time: not implementable in most cases for this reason. & Favours the processes that need less CPU-time                                                                              \\ \hline
			Shortest Remaining Time First (SRTF) & YES                 & NO                & HIGH                     & HIGH               & LONG for long processes, SHORTEST for fast processes & MEDIUM                                          & MEDIUM                                          & If a shorter task is available, it runs immediately in CPU, less overhead (the scheduler decides only when another process approaches) & Long processes may starve (worst than SJF), must estimate process time.                                 & It's the SJF but with preemption                                                                                           \\ \hline
			Higher Response Ratio Next           & NO                  & YES               & HIGH                     & MEDIUM             & MEDIUM                                               & LOW                                             & MEDIUM                                          & Prevents starvation by assigning higher priority to processes waiting for a long time.                                                 & must estimate process time                                                                              & It's the evolution of SJF.                                                                                                 \\ \hline
			Feedback                             & YES                 & NO                & HIGH                     & LOW                & LONG                                                 & LOW                                             & LOW                                             & No starving processes, don't know remaining time process needs to execute.                                                             & High overhead and continuous context switch                                                             & Penalizes jobs that have been runnning longer                                                                              \\ \hline
			Fair-Share Scheduling (FSS)          & YES                 & NO                & LOW                      & LOW                & LONG                                                 & LOW                                             & LOW                                             & Every process has its fair share of CPU time, useful with multiple group users.                                                        & Processes in crowded groups get less CPU time                                                           & Applies the round-robin scheduling strategy at each level of abstraction (processes, users, groups, etc.)                  \\ \hline
			Round Robin                          & YES                 & NO                & HIGH                     & MEDIUM             & LONG                                                 & LOW                                             & LOW                                             & No starvation, equal waiting time                                                                                                      & Longer turnaround time (every process takes longer if slower than a time quantum)                       & Uses preemption based on a clock, every process is executed in 1 time quantum and than leaves the CPU to the next process. \\ \hline
		\end{tabular}%
	}
	\caption{Main Scheduling Algorithms}
	\label{Tab:Algorithms}
\end{table}

\pagebreak

\section{Multilevel Queue}
\begin{itemize}
	\item{\textbf{multilevel queue}}, the scheduler select the queue for a process according to its type. Process does not change its queue
	\item{\textbf{multilevel feedback queue}}, process starts in the first queue but if do not complete after the first execution scheduler in the next queue
\end{itemize}

\section{Multiprocessor Scheduling}
\begin{table}[h]
	\centering

	\begin{tabular}{|m{\textwidth/15}|m{\textwidth/6}|m{\textwidth/6}|m{\textwidth/6}|m{\textwidth/6}|}
		\hline
		& \textbf{COMPLEXITY} & \textbf{OVERHEAD} & CPU UTILIZATION & ISSUES \\ \hline
		\textbf{MASTER/ SLAVE} & LOW, there is a master processor that manage the ready process queue and the common resources & LOW, static assignament of process there are few content switch & LOW, a low demanding task can lock a cpu & the master core is the bottle neck of the system\\ \hline
		
		\textbf{PEER} & HIGH, every processor has its own scheduler that have to race for common resources & HIGH, the same process can start in a processor and then jump to another & HIGH &high complexity in shared resources management \\ \hline
		
	\end{tabular}
	\caption{Multiprocessing scheduling characteristics}
	\label{tab:multiprocessing}
\end{table}

\section{Content Switch}
is a particular state of OS in which the current process in execution is stopped, the context (value of the program counter and of the register) is saved in PCB (process control bock) to allow to conitnue the execution on the future and is set the context of the new process selected from the ready queue by the scheduler.\\
\textbf{When it occours:} \\
\begin{itemize}
		\item in multitasking execution
		\item when an interrupt occours
		\item switching from kernel mode to user mode (in some architectures the cost of this operation can be lower than the other two cases)
\end{itemize}
\textbf{hardware implementation:} \\
a content switch can be implemented \underline{explicitly} with a CALL or JMP instruction to the TSS (task state segment) in global descriptor table\\
or \underline{implicitly} by hardware support that when an interrupt occours automatically save the context in special shadow registers and execute the task according to the interrupt descriptor table.\\
Sometime there can be some limitation to the hardware approach that depends from the archiecture, for example in some case the registers used by the task can be different from the general pourpose regiser of shadow reg and a software approach would be better.







