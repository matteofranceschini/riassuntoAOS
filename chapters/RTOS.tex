\section{Types}
There are two types of RTOS:
\begin{itemize}
	\item \textbf{Hard real-time}: timing violation is not accettable (car, ecc..)
	\item \textbf{Soft real-time}: timing violation is undesiderable but accettable (video, music, ecc..)
\end{itemize}

\section{Features}
The main features of an RTOS are (Un Ricordo Rovinato Definitivamente):
\begin{itemize}
	\item \textbf{Determinism}: it defines how long the operating system delays before acknowledging an interrupt and whether the system has sufficient capacity to handle all requests within required time. Operations are performed at fixed and predetermined times \textbf{or} within predetermined time intervals. It defines how much time it takes to react, but \textbf{not} to complete the operation (i.e. how long does it take to handle and interrupt). 
	\item \textbf{Resposiveness}: how long, after acknowledgement, the operating system takes to \textbf{service} (complete) the task (i.e. the interrupt). It depends on ISR nesting: if a context switch is necessary, the delay is longer than an ISR executed within the context of the current process.
	\item \textbf{Reliability}: degradation of performances should not impact the system:	
	\begin{itemize}
		\item Soft fail: Ability to fail in such a way to preserve as much data
		\item Stability: the system adopts policies such that the most critical high priority tasks execute first, even if some needs of less important tasks cannot be met.
	\end{itemize}
	\item \textbf{User control}: the user is active part of the system. He can specify task priority, choose the algorithms to use, etc.
\end{itemize}




\section{Algorithms}
\textbf{E}mpiricamente \textbf{R}icordo di \textbf{S}coreggiare - \textbf{D}ouble \textbf{P}enetration \textbf{B}ased.
%\begin{itemize}
%	\item \textbf{static-table driven}	
%	
%	
%	\item \textbf{dynamic-best effort}
%	\item \textbf{}
%\end{itemize}\
	
\begin{table}[]
	\centering
	\resizebox{\textwidth*9/10}{!}{%
		\begin{tabular}[\scriptsize]{|m{\textwidth/10}|m{\textwidth/10}|m{\textwidth/10}|m{\textwidth/10}|m{\textwidth/10}|m{\textwidth/10}|}
			\hline
			\textbf{Algorithm} & \textbf{Online Schedulability analysis} & \textbf{Priority assignament} & \textbf{Advantages} & \textbf{Issues} & \textbf{Description} \\ \hline
			
			\textbf{Static-Table Driven, (Time triggered system, as Earliest Deadline First)} & NO. Offline the condition is $\sum \frac{C_i}{T_i} < 1$  & STATIC & Easy to implement & the system can't manage tasks that aren't in the TDL. Executing a task that could not met the deadline may prevent other tasks to reach their deadlines and the number of tasks that will miss the deadline is unpredictable. & Sync of process and dependencies are saved in a TDL (Task Descriptor List). At every time-based clock, processes are executed according to this table. (applicable to periodic tasks).   \\ \hline
			
			
			\textbf{Static priority driven preemptice (Rate Monotonic)} & NO. The offline condition of schedulability is $\sum \frac{C_i}{T_i} < n(2^{\frac{1}{n}}-1)$ there is a upper limit 69 percent  & STATIC & Simpler than EDF, if the schedulability is valid, ensures we will never miss a deadline. & It's not always possible to achieve maximum CPU utilization & Every task has a priority determined statically,in RM shortest period task highest priority and therefore is executed first. \\ \hline
			
			\textbf{Dynamic Best Effort (can be based on EDF)} & NO.  & DYNAMIC & Easy, no analysis of schedulability, tipically the tasks are aperiodic and no static analysis could be possible. & Until the task is completed or the deadline arrives, there's no way of knowing if the time constraints will be met & The system assignes a priority to tasks based on their characteristics (i.e. EDF, but others can work,too). The system tries to meet dedlines and aborts any started process whose deadline is missed. \\ \hline

	
			
			\textbf{Dynamic Planning Based} & YES & DYNAMIC & A task is accepted for execution if and only if is feasible to meet its time constraints & Quite complicate to implement & Before a task execution an attempt is made to create a schedule including previous tasks and the new one. \\ \hline
			
		\end{tabular}%
	}
	\caption{RTOS algorithms}
	\label{tab:RTOS_algs}
\end{table}




\chapter{RT scheduling}

\section {cyclic scheduling}
\section{deterministic scheduling}
\section{capacity-based scheduling}
\section{dynamic priority scheduling}
\section{Scheduling Task with imprecise Results}

