%% Do not edit unless you really know what you are doing.
\documentclass[twoside,english]{report}
\usepackage[sc]{mathpazo}
\usepackage[scaled=0.9]{helvet}
\renewcommand{\ttdefault}{lmtt}
\usepackage[T1]{fontenc}
\usepackage[latin9]{inputenc}
\usepackage[a4paper]{geometry}
\geometry{verbose,lmargin=2cm,rmargin=2cm}
\usepackage{fancyhdr}
\pagestyle{fancy}
\setcounter{secnumdepth}{3}
\setcounter{tocdepth}{3}
\setlength{\parskip}{\smallskipamount}
\setlength{\parindent}{0pt}
\usepackage{babel}
\usepackage{nomencl}
% the following is useful when we have the old nomencl.sty package
\providecommand{\printnomenclature}{\printglossary}
\providecommand{\makenomenclature}{\makeglossary}
\makenomenclature
\usepackage[unicode=true,
 bookmarks=true,bookmarksnumbered=false,bookmarksopen=false,
 breaklinks=false,pdfborder={0 0 1},backref=false,colorlinks=false]
 {hyperref}
\usepackage{breakurl}

%%COMMENTARE PRIMA DI PUBBLICARE 
%----------------------------------------------------------------
\usepackage{draftwatermark} %per il watermark sullo sfondo
%----------------------------------------------------------------
% Per mettere dei commenti per revisione si possono usare i comandi:
% \unsure per le parti in cui non si � sicuri
% \change per le parti che devono essere cambiate in fase di revisone
% \improvement per le parti che devono essere descritte meglio
% \info per scrivere un commento informativo
% \thiswillnotshow � un commento che viene visto solo nel punto in cui viene piazzato e non nella lista delle note
%----------------------------------------------------------------

\makeatletter


% Customization file for the titlepage and document
%************************************************************
% Required stuff
%************************************************************
\usepackage{graphicx}
\usepackage{euler}
\usepackage[detect-all]{siunitx}
\usepackage{sectsty}
\usepackage[font={footnotesize }]{caption}
\usepackage{multicol}
\usepackage{prettyref}
\usepackage{listings}

\allsectionsfont{\rmfamily}

% Page customization
\usepackage{fancyhdr}
\pagestyle{fancy}

% Color
\usepackage{color}
\definecolor{light-gray}{gray}{0.85}
\definecolor{dark-gray}{gray}{0.75}

\fancyhead{}  % clear all header fields
\fancyhead[LO,RE]{\rule[-2ex]{0pt}{2ex}\fontsize{9}{11} \selectfont \myPhase}
\fancyhead[RO,LE]{\raisebox{-0.35cm}{\includegraphics[height=1.2cm,keepaspectratio]{gfx/Logo_TotalBlack.pdf}}}
\fancyhead[CO,CE]{\fontsize{9}{11} \selectfont \myIPT}
\fancyfoot{}  % clear all footer fields
\fancyfoot[RO,LE]{\fontsize{5.5}{8} \selectfont This work is to be considered classified. The use is allowed only for Skyward Experimental Rocketry related activities and projects. For public access please contact \href{mailto:info@skywarder.eu}{info@skywarder.eu}}
\fancyfoot[RE,LO]{\fontsize{9}{11} \selectfont \thepage}
\fancyheadoffset[LE,RO]{0.2pt}
\renewcommand{\headrulewidth}{0.2pt}
\renewcommand{\footrulewidth}{0.2pt}
\renewcommand{\headrule}{\hbox to\headwidth{%
   \leaders\hrule height \headrulewidth\hfill}}
\renewcommand{\footrule}{\hbox to\headwidth{%
    \leaders\hrule height \headrulewidth\hfill}}
\hypersetup{colorlinks=true, linkcolor=blue ,linktoc=page,citecolor=black}

\usepackage{pdfpages} %per includere pdf di più pagine


% Custom notes
\usepackage{xargs}                      % Use more than one optional parameter in a new commands
\usepackage[pdftex,dvipsnames]{xcolor}  % Coloured text etc.

\usepackage[colorinlistoftodos,prependcaption,textsize=tiny]{todonotes}
\newcommandx{\unsure}[2][1=]{\todo[linecolor=red,backgroundcolor=red!25,bordercolor=red,#1]{#2}}
\newcommandx{\change}[2][1=]{\todo[linecolor=blue,backgroundcolor=blue!25,bordercolor=blue,#1]{#2}}
\newcommandx{\info}[2][1=]{\todo[linecolor=OliveGreen,backgroundcolor=OliveGreen!25,bordercolor=OliveGreen,#1]{#2}}
\newcommandx{\improvement}[2][1=]{\todo[linecolor=Plum,backgroundcolor=Plum!25,bordercolor=Plum,#1]{#2}}
\newcommandx{\thiswillnotshow}[2][1=]{\todo[disable,#1]{#2}}
\reversemarginpar
\setlength{\marginparwidth}{2cm}
%End custom notes


%************************************************************
% Redefining numbering for sections
%************************************************************
%\renewcommand*\thesection{\arabic{section}}

%************************************************************
% Cross reference set-up
%************************************************************
\newrefformat{tab}{Table\,\ref{#1}}
\newrefformat{fig}{Figure\,\ref{#1}}
\newrefformat{eq}{Eq.\,\textup{(\ref{#1})}}
\newrefformat{sec}{Sec.\,\ref{#1}}
\newrefformat{sub}{Sec.\,\ref{#1}}

%************************************************************
% Fancy stuff
%************************************************************
\newcommand{\titlecap}[1]{\Huge{\textrm{#1}}}
\newcommand{\subtitlecap}[1]{\Large{\textsc{#1}}}
\newcommand{\sscap}[1]{\textbf{#1}}
\newcommand{\strong}[1]{\textbf{#1}}
\setlength{\headheight}{60pt} %%or

%************************************************************
% Helpful stuff to modify here, not in the LyX Document
%************************************************************
\newcommand{\myDate}{\today}
\newcommand{\myGroup}{Skyward Experimental Rocketry}
\newcommand{\myUrl}{\url{http://www.skywarder.eu}}
\newcommand{\myUni}{Politecnico di Milano}

\newcommand{\myPhase}{Riassunto globale}
\newcommand{\myProject}{Advanced Operating Systems}
\newcommand{\myIPT}{Electronic Systems\\ Integrated Project Team}
\newcommand{\myTitle}{Riassunto}
\newcommand{\myAuthor}{Matteso}
\newcommand{\myEditor}{Nobody}
\newcommand{\myEmail}{XX@skywarder.eu}

\newcommand{\mail}[1]{\href{mailto:#1}{\texttt{#1}}}

\makeatother

\begin{document}
\thispagestyle{empty}
\pdfbookmark{Titlepage}{Titlepage}

\vspace{3cm}
\begin{center}
\bigskip
\Large{\myDate}
\vspace{0.5cm}

{\titlecap{Project \myProject} \\
\vspace{0.3cm}
\titlecap{\myPhase}}\\
\vspace{0.4cm}
\rule{\linewidth}{0.5mm}
\titlecap{\myTitle}
\end{center}

\vfill
\begin{multicols}{2}
\centering{
\includegraphics[height=4cm]{gfx/Logo_Colour.pdf} \\
\includegraphics[height=3cm]{gfx/Logo_Polimi}
}
\vfill
\columnbreak
{\centering{
	\subtitlecap{\myIPT} \\
	\vspace{0.7em}
	\normalsize
	\textrm{\myGroup \\
	\myUni }}}
\vfill

\raggedright{\textbf{Author}: {\myAuthor}\\
\textbf{Editor}: {\myEditor}}

						
\end{multicols}

\clearpage

%*******************************************************
% Titleback
%*******************************************************
\thispagestyle{empty}

\hfill
\vspace{5cm}

\strong{Abstract}\\
Scrivi qui il tuo abstract
\vfill

\begin{multicols}{2}
\medskip
\noindent{\sscap{Website}}: \\
\url{http://www.skywarder.eu}


\medskip
\noindent{\sscap{E-mail}}: \\
\mail{\myEmail}
\vfill
\columnbreak
\section*{Restricted use policy}
\fontsize{8}{11} \selectfont This report is developed during the activities done within Skyward Experimental Rocketry association. Its use is allowed only for Skyward Experimental Rocketry related purposes. If you're a Skyward member, please don't send or release publicly this file without previous acceptance from Direction Board.
For public access and publication please contact \href{mailto:info@skywarder.eu}{info@skywarder.eu}.
\end{multicols}
\vspace{1cm}
\hrule
\bigskip
\clearpage


\pagenumbering{roman}

\begin{multicols}{2}

\printnomenclature{}

\end{multicols}

\tableofcontents{}

\listoffigures


\listoftables

%%COMMENTARE PRIMA DI PUBBLICARE 
%----------------------------------------------------------------
\listoftodos[Notes] %per avere l'elenco delle note
%----------------------------------------------------------------

\clearpage{}

\pagenumbering{arabic}

\setcounter{page}{1}

\global\long\def\diff{\text{d}}



\chapter{GitHub}
\section{initialize repository}
\begin{itemize}
\item \textbf{git init}, create a new folder with files 
\item \textbf{git add $<files>$}, with "." add every file in the folder, "*.tex" just the \LaTeX   files.
\item \textbf{git commit .}, commits every file in the folder.
\item \textbf{git remote add $<repo name>$ $<repo url>$},create a new repo on github website
\item \textbf{git push -u $<repo name>$  $<branch name>$}
\end {itemize}


\section{work on repository}
\begin{itemize}
\item \textbf{git add} $<file to add to staging area>$
\item \textbf{git commit -m "comment" }, commits locally the changes, adding an inline comment.
\item \textbf{git push -u$<repo name>$ $<branch name>$}, the "-u" is needed only if you want to commit from a specific user.
\end{itemize}

\section{remote repository management}
\begin{itemize}
\item \textbf{git clone $<repository\ url>$} downloads the full remote repository to the local computer, but has to be done on an empty folder.
\item \textbf{git fetch $<repository\ name>$}, downloads the full remote repository to the local computer, overwriting the local files.
\item \textbf{git pull $<repository\ name>$ $[branch\ name]$}, downloads the changes from the remote repository.
\item \textbf{git push -u $<repository\ name>$ $[branch\ name]$}, uploads the changes to the remote repository.
\end{itemize}

\section{Merge with Master}
\begin{itemize}
	\item \textbf{git merge $<branch\ name>$}, merges the specified branch with the master
\end{itemize}

\section{General use}
\begin{itemize}
	\item \textbf{git status}, gives information about the changes in the local folder.
	\item \textbf{git log}, gives information about all the commits of the repository, with each comment.
\end{itemize}

\section{branch}
\subsection{create a new branch}
\begin{itemize}
\item \textbf{git checkout -b $<new\ branch\ name>$}
\item \textbf{git stauts}
\item \textbf{git add $<file\ to\ commit>$}
\item \textbf{git commit -m "comment"}
\item \textbf{git push -u $<repo\ name>$ $<new\ branch\ name>$}
\end{itemize}


\subsection{switch to an other branch}
\begin{itemize}
\item \textbf{git checkout $<branch\ name>$}
\end{itemize}





\begin{figure}[h]
	\centering
	\includegraphics[width=\textwidth]{gfx/git_graph}
	
	\caption{Comandi GIT}
	\label{Fig:Commands}
\end{figure}

%inserire un PDF nel pdf
%\begin{figure}[h]
%	\centering
%	\includegraphics[scale=1]{doc/nome_del_documento}
%	
%	\caption{descrizione, se necessaria}
%	\label{Fig:etichetta_documento}
%\end{figure}

\end{document}
